\documentclass{jps-cp}
\usepackage{txfonts} %Please comment out this line unless the txfonts package is availabe in your LaTeX system.
\usepackage{url}

\title{New analysis method of TPC data using neural network}

\author{
  Takanobu \textsc{Doi}$^{1}$, Takahiro \textsc{Kawabata}$^{2}$, Tatsuya \textsc{Furuno}$^{3}$,
  Yuki \textsc{Fujikawa}$^{1}$, Kento \textsc{Inaba}$^{1}$, Motoki \textsc{Murata}$^{3}$,
  Shintaro \textsc{Okamoto}$^{1}$ and Akane \textsc{Sakaue}$^{1}$}

\inst{
  $^{1}$Department of Physics, Kyoto University, Kyoto, Kyoto 606-8502, Japan \\
  $^{2}$Department of Physics, Osaka University, Toyonaka, Osaka 540-0043, Japan \\
  $^{3}$Research Center for Nuclear Physics, Osaka University, Ibaraki, Osaka 567-0047, Japan }

\email{doi.takanobu.68x@st.kyoto-u.ac.jp}

\recdate{} % Write received date here

\abst{} % Write abstract here

\kword{neural networks, TPC, MAIKo} % Write keywords here

\begin{document}
\maketitle

\section{Introduction}
荷電粒子の飛跡を3次元的に検出することができる検出器としてTime Projection Chamber (TPC) がある。
検出器中にガスを従弟んすることで荷電粒子が通過した飛席上に電離が起きる。
検出面に向かって電子がドリフトするように電場をかけることで、電離した電子を検出面まで移動させる。
移動してきた電子のドリフト時間を測定することでドリフト方向の位置を決定することができる。
また、検出面に飛跡を射影した像を決定することができる。
近年、標的の薄膜を透過することができないほど低エネルギーの荷電粒子の検出に対して、
検出ガスを標的とするアクティブ標的を使用することでそのような粒子が検出できる。
しかし、TPC とアクティブ標的を用いた手法は背景事象となるイベントの除去や
散乱粒子及び反跳粒子の飛跡情報の抽出などが容易ではなく解析に大きな労力を要する。

我々はMicro Pixel Chamber ($\mu$-PIC)~\cite{mupic} で読み出しを行うTPC とアクティブ標的を用いた検出器である
Mu-Pic based Active target for Inverse Kinematics . (MAIKo) TPC~\cite{MAIKo}を不安定核実験に用いるために開発した。
近年、大阪大学核物理研究センター (RCNP) において、
MAIKo TPC を用いた${}^{10}\rm{C}$と${}^{4}\rm{He}$の非弾性散乱の測定が初めて行われた。
MAIKo TPC から得られるデータから背景事象の除去をこれまではHough 変換による飛跡抽出アルゴリズムを用いて行ってきた。
しかし、この手法にはアルゴリズムに多くのパラメータの最適化とその後の画像識別に多くの時間を必要とする。
また、識別能も満足のいくものではなかった。
そこで、我々は近年注目されているニューラルネットワークを用いた画像データの識別を試みた。
ニューラルネットワークを用いることで、
離れた点の関係や直線の位置、角度などの従来の解析手法ではまとめて扱うことの難しい
多くの特徴量を同時に考慮した画像識別が可能になると期待される。
また、ニューラルネットワークは一度構築するとその後は短時間で画像識別を行うことができる。
このようなニューラルネットワークの特徴を活かすことで、
従来のアルゴリズムでは実現が難しかった高い精度と短い識別時間を実現することが可能である。

\section{Conventional analysis method}

\section{New analysis method}

\section{Result}

\section{Conclusion}



\begin{thebibliography}{9}
\bibitem{maiko}
  T.~Furuno, T.~Kawabata, H.~Ong, S.~Adachi, Y.~Ayyad, T.~Baba, Y.~Fujikawa, T.~Hashimoto, K.~Inaba, Y.~Ishii,
  S.~Kabuki, H.~Kubo, Y.~Matsuda, Y.~Matsuoka, T.~Mizumoto, T.~Morimoto, M.~Murata, T.~Sawano, T.~Suzuki, A.~Takada,
  J.~Tanaka, I.~Tanihata, T.~Tanimori, D.~Tran, M-, Tsumura, and H.~Watanabe
  Nuclear Instruments and Methods in Physics Research Section A \textbf{908} 215 (2018)
\bibitem{tensorflow}
  A.~Davis, J.~Dean, M.~Devin, S.~Ghemawat, I.~Goodfellow, A.~Harp, G.~Irving,
  M.~Isard, Y.~Jia, R.~Jozefowicz, L.~Kaiser, M.~Kudlur, J.~Levenberg,
  D.~Man\'{e}, R.~Monga, S.~Moore, D.~Murray, C.~Olah, M.~Schuster, J.~Shlens,
  B.~Steiner, I.~Sutskever, K.~Talwar, P.~Tucker, V.~Vanhoucke, V.~Vasudevan,
  F.~Vi\`{e}gas, O.~Vinyals, P.~Warden, M.~Wattenberg, M.~Wicke, Y.~Yu, and
  X.~Zheng, {TensorFlow: Large-Scale Machine Learning on Heterogeneous Systems} (2015).
  \url{https://tensorflow.org}
\bibitem{keras}
  F.~Chollet, et al. {Keras} (2015). \url{https://keras.io}

\bibitem{cp} The abbreviation for JPS Conference Proceedings should be ``JPS Conf. Proc." in the reference list.
\bibitem{jpsj} The abbreviation for the Journal of the Physical Society of Japan should be ``J. Phys. Soc. Jpn." in the reference list.
\bibitem{ptep} The abbreviation for the Progress of Theoretical and Experimental Physics should be ``Prog. Theor. Exp. Phys." in the reference list.
\bibitem{instructions} More abbreviations of journal titles are listed in ``Instructions for Preparation of Manuscript", which is available at our Web site (http://jpsj.jps.or.jp).
\bibitem{format} F. Author, S. Author, and T. Author, Abbreviated journal title \textbf{volume in bold face}, initial page or article number (year of publication).
\end{thebibliography}

\end{document}

